\documentclass[margin,line]{res}
\usepackage{hyperref}
\usepackage{url}
\usepackage{color}
\usepackage[official]{eurosym}
\hypersetup{hidelinks=true,colorlinks=true,citecolor=red}
\oddsidemargin -.5in
\evensidemargin -.5in
\textwidth=6.0in
\itemsep=0in
\parsep=0in
\topmargin=0in
\topskip=0in
 
\newenvironment{list1}{
  \begin{list}{\ding{113}}{%
      \setlength{\itemsep}{0in}
      \setlength{\parsep}{0in} \setlength{\parskip}{0in}
      \setlength{\topsep}{0in} \setlength{\partopsep}{0in}
      \setlength{\leftmargin}{0.17in}}}{\end{list}}
\newenvironment{list2}{
  \begin{list}{$\bullet$}{%
      \setlength{\itemsep}{0in}
      \setlength{\parsep}{0in} \setlength{\parskip}{0in}
      \setlength{\topsep}{0in} \setlength{\partopsep}{0in}
      \setlength{\leftmargin}{0.2in}}}{\end{list}}
    
\begin{document}
\name{\LARGE Madhavan Pallan} %\hfill {}

%{\em \today}

\begin{resume}
% \url{http://madhavanpallan.github.io}
\section{\sc Contact Information}

\vspace{.05in}
\begin{tabular}{@{}p{3.5in}p{3in}}     
{E-mail:}  madhpallan@gmail.com & {Phone:}  (+91) 8800996753   \\
{Website:} \url{http://madhavanpallan.github.io} & {City:} New Delhi, India\\
\end{tabular}

%\begin{tabular}{@{}p{3.5in}p{3in}}     
%	TF02, IBM India Research Lab.  & {Phone:}  (+91) 8800996753 \\
%	Plot No.4, Phase II, Block - C, ISID Campus.  & {E-mail:}  madhpallan@gmail.com\\
%	Institutional Area, Vasant Kunj. &  {Website:}\\
%	New Delhi - 110070,India.  &
%	%{Website:} \url{https://sites.google.com/site/madhavanpallan1210/} 
%	\url{http://madhavanpallan.github.io}
%\end{tabular}

\section{\sc Interests}
Machine Learning, Large Scale Data Mining, Deep Learning, Social Network Analysis, Artificial Intelligence. {\bf Key Strengths also include} Information Extraction from web, Data Analytics for large Data, Data Visualization, JUNG, SimMetrics, JSON . %In {\bf Image Processing} Image Extraction, Filtering, Image Extraction.

\section{\sc Education}
{\bf Indian Institute of Technology (BHU) - Varanasi} \hfill July 2008 -- May 2013\\
%\vspace*{-.1in}
Five Year Integrated Masters Degree - Mathematics and Computing \hfill(DGPA 8.08/10.0){\bf(Honors)}%(Absolute Scale) \\
%{\bf Vivek Junior College}, Mumbai, Maharashtra. India \hfill August 2005 -- August 2007\\
%Higher Secondary Certificate \hfill(GPA 72/100)

%%%%%%
\section{\sc Professional Experience}
%%%%%%
{\bf {\em IBM India Research Lab}, New Delhi. India.} \hfill{July 2013 -- Present}\\
\href{http://researcher.ibm.com/researcher/view.php?person=in-mapallan}{\em {\bf Research Engineer}}
\vspace{2.0 mm}\\
%\hfill {Summer 2011, June 2011 -- July 2011}\\
%%%%%%
%\begin{itemize}
%  \setlength{\itemsep}{0.01pt}
%  \setlength{\parskip}{0pt}
%  \setlength{\parsep}{0pt}
%%	\item Interacts with top notch Researchers from the fields of Machine Learning, Statistics, Optimization, and Artificial Intelligence to bring about constructive output in a project.
%%  	\item Provide key suggestion in development and research.
%%	\item 
%	\item Developed DataBase Schema in IBM DB2, Visualization of Data using IBM Cognos and D3 for Demonstrating an Enterprise Project Predictive Talent Mobility, which bring scholastic analytics to Hiring Committee.
%	\item Part of Development Team Smarter Planet, and developed City-concierge and Bharat-Khoj.
%	\item The Current Project aims at controlling the flow of traffic using Megaffic a simulation tool used to study Traffic.
%\end{itemize}
{\bf{\em Google Summer of Code - Scilab}}\hfill {May 2012 -- September-2012}\\
\href{https://research.google.com/university/student-support/}{\em {\bf Research Program}}
%
%%%%%%
\vspace{0.01 mm}
% {\bf{\em Advant Technologies and Inc.}}\hfill {May 2011 -- June -2011}\\
% {\em Summer Intern}
% %%%%%%%%%%%
% \vspace{1.4 mm}
%\section{\sc Awards and Fellowships}
%\begin{itemize} %Job Description%
%\item Received {\bf IIT(BHU)-Color 2012}, highest award for outstanding contribution in the field of Robotics
%% apart from Regular Curriculum.
%\item Awarded {\bf Western Railway Union Medal 2013}, for being in top 5 students in a category given by Western Railway, Indian Government.
%\item Recipient of {\bf MHRD Scholarship, 2012 - 13} for good academic performances.
%\item Recognized by {\bf Student Leadership Award 2012}, for showing exemplary leadership skills in conducting and organising workshop
%\item Stood {\bf Third} in a SANRACHNA a model building competition in Departmental Fest Anveshan 2012.
%\item Recipient of {\bf Western Railway Scholarship 2010 - 12}, for maintaining SGPA of 8 for both the years.
%\item Received {\bf Merit-cum-mean Scholarship 2008 - 10 :}, given to top 25\% Undergraduate meritorious student in whole of IIT.
%\end{itemize}
\section{\sc Publications}
\begin{itemize}
\item[-]S Vasudevan, R Chaudhuri, M Pallan, S S Singh{\bf ``An Empirical Study of Starting Salaries and Employment Trends of Engineering Students in India"},{\em In Proceedings of the Sixth Volume of Journal of Data Science}, 2017 {\bf \href{http://epubs.siam.org/doi/pdf/10.1137/1.9781611973440.102}{\em (JDS'17)}}

\item[-] M Pallan, B Srivastava, {\bf ``Automatically Augmenting Titles of Research Paper for Better Discovery"},{\em In Proceedings of the Thirtieth AAAI Conference on Artificial Intelligence at AAAI Conference}, 2016{\bf \href{https://www.aaai.org/ocs/index.php/WS/AAAIW16/paper/view/12578/12441}{\em (AAAI'16)}}

\item[-] B Srivastava, M Pallan, M Madhavan, R Kokku, {\bf ``Case Studies in Managing Traffic in a Developing Country with Privacy-Preserving Simulation as a Service"},{\em In Proceedings of the IEEE International Conference on Services Computing (SCC) (2016) at SCC Conference}, 2016{\bf \href{https://www.computer.org/csdl/proceedings/scc/2016/2628/00/2628a766-abs.html}{\em (SCC'16)}}

\item[-] A Bhowmik, V Borkar, D Garg, M Pallan, {\bf ``Submodularity in Team Formation"},{\em In Proceedings of the Fourteenth SIAM International Conference on Data Mining at SDM Conference}, 2014 {\bf \href{http://epubs.siam.org/doi/pdf/10.1137/1.9781611973440.102}{\em (SDM'14)}}




\end{itemize}
\section{\sc Awards and Fellowships}
\begin{itemize}
\item[-]Recipient of {\bf Eminence and Excellence Award} for leading a research project, which was ranked 1st position globally all over IBM Research, Inside Analytics - The same project received highest number of views in IBM Next for a good period of time. (IBM Next-Restricted access to internal researchers only!!) -12th Feb 2016.
\item[-]Recipient of Manager Choice Award from the Group Business Analytics and Mathematical Sciences (BAMS) - 5th June 2015.
\item[-]Part of IBM Research Half of Fame for projects involved in CitySDK with Senior Researcher and IBM Next Project 2016, for the period 2014-16.
\item[-]{\bf Runner up in CitySDK Smart Participation Challenge}, received EU wide Recognition.
\item[-]Awarded {\bf Third} position in a {\bf IBM Consumer Analytics Competition 2014} among IBM Research Global.
\item[-]Awarded {\bf First} position out of thirty {\bf Data Challenge} in ACM IKDD (Indian version of KDD) in the Conference of Data Science.
% apart from Regular Curriculum.
\item[-]Awarded {\bf Western Railway Union Medal 2013}, for being in top 5 students in educational category given by Western Railway, Indian Government.
\item[-]Received {\bf IIT(BHU)-Color 2012}, highest award given for outstanding contribution in the field of Robotics, by Science and Technology Council @IIT(BHU), India. %the field of
%\item[-]Recipient of {\bf MHRD Scholarship, 2012-13} for overall academic performances.
\item[-]Recognized by {\bf Student Leadership Award 2012}, for showing exemplary leadership skills. % in conducting and organizing workshop.
\item[-]Awarded {\bf Third} position in SANRACHNA a model building competition in Departmental Fest {\bf Anveshan} 2011.
\item[-]Recipient of {\bf Western Railway Scholarship 2010-12}, for maintaining {\bf First Class} in GPA.
%\item[-]Received {\bf Merit-cum-mean Scholarship 2008-10}, given to top 25\% Undergraduate meritorious student in whole of IIT.
\end{itemize}
% {\bf IIT(BHU)-Color 2013 :} A prestigious award, received for showing outstanding and significant contribution in field of Robotics. This is mainly for conducting and organizing workshops and competitions to bring about awareness and technical development in Students.

% {\bf Western Railway Union Medal 2013 :} This medal was given to top 5 Students who have shown exemplary performance in their field of Engineering, and are childrens of Western Railway 
% Employee's. The Award also included a cash Prize of 5000 INR.
%{\bf IIT(BHU) Student LeaderShip Award 2012 :} This was given for demonstrating leadership skills among the IIT(BHU) Students in the academic year 2011-12. The achievements also include Organizing and Conducting Preliminary Robotics workshop, Technical Head of Fest Technex -2012
%{\bf Master Scholarship 2012 - 2013 :} This stipend is received by the Students ,who have cleared Graduate Aptitude Test in Engineering(GATE), and are Pursuing masters in the field of Engineering.
% {\bf Western Railway Scholarship 2010 - 2012 :} This annual scholarship is given to student of western railway employee's, who maintain a minimum of first class at the end of every year.
% {\bf Merit-cum-mean Scholarship 2008 - 2010 :} This is given to 25\% Undergraduate meritorious student in whole of IIT.


%   {\bf IIT(BHU)-Color 2013 :} A prestigious award, received for showing outstanding and significant contribution in field of Robotics. This is mainly for conducting and organizing workshops and competitions to bring about awareness and technical development in Students.
% \vspace{1.4 mm}\\
% {\bf Western Railway Union Medal 2013 :} This medal was given to top 5 Students who have shown exemplary performance in their field of Engineering, and are childrens of Western Railway Employee's. The Award also included a cash Prize of 5000 INR.
% \vspace{2.0 mm}\\
% {\bf IIT(BHU) Student LeaderShip Award 2012 :} This was given for demonstrating leadership skills among the IIT(BHU) Students in the academic year 2011-12. The achievements also include Organizing and Conducting Preliminary Robotics workshop, Technical Head of Fest Technex -2012
% \vspace{2.0 mm}\\
% {\bf Master Scholarship 2012 - 2013 :} This stipend is received by the Students ,who have cleared Graduate Aptitude Test in Engineering(GATE), and are Pursuing masters in the field of Engineering.
% \vspace{2.0 mm}\\
% {\bf Western Railway Scholarship 2010 - 2012 :} This annual scholarship is given to student of western railway employee's, who maintain a minimum of first class at the end of every year.
% \vspace{2.0 mm}\\
% {\bf Merit-cum-mean Scholarship 2007 - 2009 :} This is given to 25\% Undergraduate meritorious student in whole of IIT.

%%%%%%%%%%%%%%%%%%%

%%%%%%%%%%%%%%%%

\section{\sc Projects}
%%%%%
%\begin{itemize}
%  \setlength{\itemsep}{0.01pt}
%  \setlength{\parskip}{0pt}
%  \setlength{\parsep}{0pt}
%	\item Interacts with top notch Researchers from the fields of Machine Learning, Statistics, Optimization, and Artificial Intelligence to bring about constructive output in a project.
%  	\item Provide key suggestion in development and research.
%	\item Developed DataBase Schema in IBM DB2, Visualization of Data using IBM Cognos and D3 for Demonstrating an Enterprise Project Predictive Talent Mobility, which bring scholastic analytics to Hiring Committee.
%	\item The Current Project aims at controlling the flow of traffic using Megaffic a simulation tool used to study Traffic.
%\end{itemize}

%%%%%%
% {\bf Predictive Talent Mobility} \hfill {January 2014 - present}
% \begin{itemize} %Job Description%
%   \setlength{\itemsep}{0.01pt}
%   \setlength{\parskip}{0pt}
%   \setlength{\parsep}{0pt}
% \item PTM short for Predictive talent Mobility is an Enterprise Project under the Business Analytics and Mathematical Sciences (BAMS) group at IBM IRL. It eases the problems of recruitment by predicting mobility of the Talented candidate in the large pool of employees in an organization by calculating technical math score, interview clearing likelihood score and onboarding score. 
% \item The Product is a combination of three views namely Employee, Hiring Manager and HR Manager View. Each view performs its own analytics to show the best of data provided.
% \item Developed Database schema in IBM DB2, and D3 an opensource tool for the end product visualization for this large Scale data. 
% \end{itemize}
% \vspace{7.0 mm}

%{\bf Automatically Augmenting Titles of Research Paper for Better Discovery} \hfill {June 2014 - present}
%\begin{itemize} %Job Description%
%  \setlength{\itemsep}{0.01pt}
%  \setlength{\parskip}{0pt}
%  \setlength{\parsep}{0pt}
%\item Team: Dr Biplav Srivastava
%\item Investigated the characteristics of titles of AI and CS conference papers and then propose automatic ways to augment them so that they can be better indexed and discovered by users. 
%\item Evaluate relationship between title lengths of CS papers with citations for well-established papers, and found that it follows sinusoidal function, the highest R-square value for this fit was 89.07 \%, with the 5\% RMSE \% Error. And Prescribed Augmentation Method.
%%\item Prescribe augmentation of titles with additional metadata that included two different augmentation approaches - session names for given conference and ACM CCS.
%\item Best augmentation was done by levenshtein and cosine similarity score for the given title with the session name and ACM CCS, and was further assisted by User Study. 
%%\item A User study was conducted, in which people preferred titles augmented with ACM CCS over Session Names and only Titles.
%\end{itemize}
%\vspace{0.8 mm}

% Predictive Talent Mobility, also known as PTM is a initiative by the Bussiness Analytics and Mathematics Science group to give a demonstration of the Predictive Analysis involved in Recruitment of a candidate for a job. There are three views namely Employee view, Hiring Manager View and HR Manager View. Depending on the type of role a person logins into the web Application and looks for the various data. For e.g. A Employee get to look at all the jobs he can apply or applied. A Hiring Manager can see the number of candidates who have applied for the job and finally, HR Manager can have a complete view of all the Jobs and the Candidates with the necessary analytics to have a better view of the project. This also includes integration of d3 technology in web-application. This project is externally available.

% {\bf City Concierge}
% \begin{itemize} %Job Description%
%   \setlength{\itemsep}{0.01pt}
%   \setlength{\parskip}{0pt}
%   \setlength{\parsep}{0pt}
% \item Developed a smarter planet application 

% % In this application, we allow a person to know more about cities he lives in or wants to visit, provided the cities support
% % Open 311/ CitySDK data and API services. The user can use the information to compare cities based on events that are happening, the services they support and the service requests they may have pending so that he can decide which city he wants to visit and have a pleasant experience.
% \end{itemize}
%\section{\sc Personal Achievements}
{\bf City Concierge} 
\href{https://drive.google.com/file/d/0B3v1C2BWVcCRRm1VQXVOZkI4ZE0/view?usp=sharing}{\em {\bf (PPT)}} \hfill {April - June 2014}
\begin{itemize} %Job Description%
  \setlength{\itemsep}{0.01pt}
  \setlength{\parskip}{0pt}
  \setlength{\parsep}{0pt}
  \item Solved the problem of a tourism guide, which helps the travellers by showing the events with less number of issues for the nations having api following open311 standard, By applying a clustering algorithm, which looks on the various issues and list the issues for that region. This was carried under the Umbrella of Group Smarter Planet.
  \item The problem was solved by developing  Real world applications {\bf City-concierge} for EU nations (amsterdam, lisbon, helsinki and Lamia) and \href{http://bharat-khoj-starter.mybluemix.net/}{\em {\bf Bharat-Khoj}}, under the mentorship of {\bf Dr. Biplav Srivastava}, Master Inventor, ACM Distinguished Scientist, IBM India Research Lab.
  \item  \href{http://dev.hel.fi/node/191}{\em {\bf Runner up}} in CitySDK Challenge Europian Union and bagged a sum of 2000 {\bf Euro} for developing the {\bf Best Cross-City Smart Participation Application}.
  \item {\bf Patents : 1 (Under Process)}
\end{itemize}
\vspace{0.8 mm}

{\bf IBM Consumer Analytics Competition 2014} \hfill {April - May 2014}
\begin{itemize} %Job Description%
  \setlength{\itemsep}{0.01pt}
  \setlength{\parskip}{0pt}
  \setlength{\parsep}{0pt}
  \item {\bf Third} in IBM internal Machine Learning Competition, having teams participated from all Geographical Research Labs of IBM. 
  \item Under the mentor-ship of Dr. Biplav Srivastava, Master Inventor, ACM Fellow, IBM India Research Lab.
  %\item {\bf Tools used:} such as R, weka and IBM Cognos were used for the analysis of the dataset, 
  \item {\bf Methods:} Expected Maximization, Clustering and Correlation.
\end{itemize}
\vspace{0.8 mm}
% {\bf Megaffic - Mega Traffic Simulator} 
% \begin{itemize} %Job Description%
%   \setlength{\itemsep}{0.01pt}
%   \setlength{\parskip}{0pt}
%   \setlength{\parsep}{0pt}
%   \item Third in IBM internal Machine Learning Competition, having teams participated from all Geographical Research Labs of IBM. 
%   \item {\bf Tools used:} such as R, weka and IBM Cognos were used for the analysis of the dataset, 
%   \item {\bf Methods:} Expected Maximization, Clustering and 
% \end{itemize}
{\bf Submodularity in Team Formation} \hfill {July 2013 - October 2013}
\begin{itemize} %Job Description%
  \setlength{\itemsep}{0.01pt}
  \setlength{\parskip}{0pt}
  \setlength{\parsep}{0pt}
\item Team: Avradeep Bhowmik, Dr V S Borkar, Dr Dinesh Garg
\item The problem of finding an optimal team which satisfies the constraints of having a team with certain skill requirement and social compatability can be formulated as a 
submodular maximization problem, and discovered that the submodular function turns out to be non-negative and non-montone.
\item Here, we used Submodular maximizartion by simulated annealing for this problem to obtain a 0.41-approximation
%\item Using the submodular structure helped in posing skill as soft requirement. We used Jaccard distance to measure the social compataility of two individuals on the given 
Social graph created from DBLP Dataset.
\item Experimental study of Average team size, Average Number of missing skills and Average Number of Connected Components was calculated for different number of required skills for team formation.
\end{itemize}
\vspace{0.8 mm}
{\bf Integration of Leptonica library in Siptoolbox (M-Tech Thesis)}\hfill {January 2012 -- May-2013}
%{\bf Leptonica library made available in Scilab (M-Tech Thesis)}\hfill {Dr. Tanmoy Som and Dr. Ricardo Fabbri}
\begin{itemize} %Job Description%
  \setlength{\itemsep}{0.01pt}
  \setlength{\parskip}{0pt}
  \setlength{\parsep}{0pt}
\item It Comprises of the challenges like designing and developing intermediate functions encountered in making leptonica library available to Scilab through siptoolbox (Scilab Image Processing ToolBox). And is an extension of work done in Google Summer of Code'12, under my Thesis Advisor Dr Tanmoy Som and Dr Ricardo Fabbri.
\item Description of the functions like deskew and dewarp, having state of the art algorithm found in leptonica were now made available through siptoolbox
\item Extensive testing of the made available functions were done on different types of images like, jpeg, png and tmp. Complete change called \href{http://www.google-melange.com/gsoc/proposal/public/google/gsoc2012/mady902/5668600916475904}{\em {\bf (sip-gsoc)}}is now part of {\bf SIP ver 0.10.0.}
\end{itemize}
% This thesis is a composition of the various challenges encountered in making the leptonica library available to Scilab. The thesis is also a extension of the work done in Google summer of Code 2012 with the Organisation Scilab. This involved writing intermediate functions which form the fundamental background for working of both the different data-structures in scilab and leptonica. The library was made available to Scilab using siptoolbox. It also, involved making available two important leptonica functions in Scilab such as deskew and dewarp. The example functions were extensively tested on all types of images.
{\bf {\em Google Summer of Code (Internship)}}\href{http://www.google-melange.com/gsoc/proposal/public/google/gsoc2012/mady902/5668600916475904}{\em {\bf (Proposal)}}
\href{http://www.google-melange.com/gsoc/project/details/google/gsoc2012/mady902/5741031244955648}{\em {\bf (Abstract)}}
\hfill {May 2012 -- September-2012}
\begin{itemize} %Job Description%
  \setlength{\itemsep}{0.01pt}
  \setlength{\parskip}{0pt}
  \setlength{\parsep}{0pt}
%\item As part of this  to help develop large Scale OpenSource Organization. I was working with , a free and open source software for numerical computation providing a powerful computing environment for engineering and scientific applications. I was also closely working with {\bf \#labmacambira} at irc.freenode.net from October 2011 to September 2012
\item The Research Program aimed at making Google-books \href{https://code.google.com/p/leptonica/}{\em leptonica} library available to \href{http://www.scilab.org/scilab/about}{\em Scilab} a free and open source software for numerical computation providing a powerful computing environment for engineering and scientific applications through \href{http://siptoolbox.sourceforge.net/}{\em SIP ToolBox} (Scilab Image Processing ToolBox). Under the mentorship of \href{http://www.lems.brown.edu/~rfabbri/}{\em {\bf Dr. Riccardo Fabbri}}  from Rio-de-Janeiro State University. I was also closely working with {\bf \#labmacambira} at irc.freenode.net from October 2011 to September 2012
%\item An international Program, Google Summer Of Code funded by google for the development of large scale Open Source projects. 
\item Challenges involved creating Porting Functions which made leptonica functions available in siptoolbox. As, part of the verification two functions dewarp and deskew which are exclusively available in leptonica, were made available in Scilab. 
\item IIT-BHU was Ranked {\bf 6th} for number of accepted students in 2012 , and {\bf 10th} in \#students accepted from 2005 - 2012.
\item \href{http://wiki.scilab.org/Contributor-Pallan-Madhavan-GSOC2012}{\em {\bf (Detailed Timeline)}}
\href{http://www.scilab.org/projects/gsoc/2012/201211292}{\em {\bf (Final Review)}}
\href{http://vimeo.com/47705781}{\em {\bf (Deskew and Dewarp for all images)}}
\href{http://vimeo.com/47475970}{\em {\bf (Deskew Demo)}}
\href{http://vimeo.com/47705325}{\em {\bf (Dewarp Demo)}}
%http://google-opensource.blogspot.in/2012/05/google-summer-of-code-2012-stats-part-2.html
%\item Closely worked with two opensource Organizations {\bf \#scilab } at irc.oftc.net and {\bf \#labmacambira} at irc.freenode.net from October 2011 to September 2012
% \item Tools used: 
\end{itemize}
\vspace{0.8 mm}
{\bf Line Follower Robot}\hfill {May 2011 -- June -2011}
\begin{itemize} %Job Description%
  \setlength{\itemsep}{0.01pt}
  \setlength{\parskip}{0pt}
  \setlength{\parsep}{0pt}
\item Developed a algorithm for the problem of traversing on white lines on completely black painted surface and performed a task of collecting blocks from the intersection point of the 4x4 grids to the end point of the grid. with the help of 4 wheeled bot and Atmega 32.
%\item Built a 4 wheeled Robot, and programmed the Micro-controller Atmega-32 to complete the task.
\item Participated at Techniche, Technical Fest of IIT-Guwahati.
%\item {\bf Tools used:} CvAVR, AVR USB Programmer.

% \item Prices of the products, were made available to the customers automatically while billing. Calculating taxes, discount on the product was done  and  
% we made store prices available to the product purchased by the customer by quering from the database of the product. Also, doing basic calculations like VAT, service tax and discount calculations create a recepipt. This project also included interfacing to a e-receipt printer, which would finally print the bill. \\
% \item Tools used: 
\end{itemize}

{\bf Smart Furnace Alerter}\hfill {May 2011 -- June -2011}
\begin{itemize} %Job Description%
  \setlength{\itemsep}{0.01pt}
  \setlength{\parskip}{0pt}
  \setlength{\parsep}{0pt}
\item Developed a program, solving real world problem that regulates the temperature of the Metal Furnace, taking the reading from the furnace with the help of thermostat and providing readings to the micro-controller. The reading was then used to alert the workers working in factory with the help of a buzzer and displaying alert on 16x2 LCD.
%\item Teamed with Er. Manikandan Mariyappan of the Metallurgical Department.
\item Stood {\bf Third position in SANRACHNA}, a model building event in the Metallurgical Departmental fest {\bf Anveshan}.
%\item {\bf Tools used:} CvAVR.
%\item {\bf Hardware used:} RS 232 Programming interface, LCD, Thermostat.
% \item Prices of the products, were made available to the customers automatically while billing. Calculating taxes, discount on the product was done  and  
% we made store prices available to the product purchased by the customer by quering from the database of the product. Also, doing basic calculations like VAT, service tax and discount calculations create a recepipt. This project also included interfacing to a e-receipt printer, which would finally print the bill. \\
% \item Tools used: 
\end{itemize}

%{\em Image Processing Robot}\hfill {May 2011 -- June -2011}
%\begin{itemize} %Job Description%
%  \setlength{\itemsep}{0.01pt}
%  \setlength{\parskip}{0pt}
%  \setlength{\parsep}{0pt}
%\item Developed a algorithm, that detects a 
%
%to windows Application for demonstrating Billing of products using a Scanner, later integrated in an Enterprise Application
%\item Carried under the supervision of Project Manager {\bf Mrs. Deeksha Hegde}. 
%\item Tools used: Matlab
%\item RS 232 Programming interface, 4 wheeled robot.
%% \item Prices of the products, were made available to the customers automatically while billing. Calculating taxes, discount on the product was done  and  
%% we made store prices available to the product purchased by the customer by quering from the database of the product. Also, doing basic calculations like VAT, service tax and discount calculations create a recepipt. This project also included interfacing to a e-receipt printer, which would finally print the bill. \\
%% \item Tools used: 
%\end{itemize}
% {\em Sale Invoice for Retail Point of Sale}\hfill {May 2011 -- June -2011}
% \begin{itemize} %Job Description%
%   \setlength{\itemsep}{0.01pt}
%   \setlength{\parskip}{0pt}
%   \setlength{\parsep}{0pt}
% \item Developed a windows Application for demonstrating Billing system of products using a Scanner, later integrated in an Enterprise Application
% \item Carried under the supervision of Project Manager {\bf Mrs. Deeksha Hegde}. 
% % \item Prices of the products, were made available to the customers automatically while billing. Calculating taxes, discount on the product was done  and  
% % we made store prices available to the product purchased by the customer by quering from the database of the product. Also, doing basic calculations like VAT, service tax and discount calculations create a recepipt. This project also included interfacing to a e-receipt printer, which would finally print the bill. \\
% % \item Tools used: 
% \end{itemize}
% {\bf Numerical Method library}\hfill {VII Semester Project}\\
% This library was developed as part of the Curriculam. The library included operations on Matrices and Linear Systems of Equations.
% The whole of library was developed in C. The library was envisioned for numerical applications in current Scientific Problems.

% {\bf Scientific Calculator}\hfill {V Semester Project}\\
% In this project, a Scientific calculator was developed in Java. This Project had mostly all the functioanlities of a 
% basic scientific calculator.
\section{\sc Technical Skills}
{\bf Programming and Scripting Languages}:  C, CPP, Java, {\bf Familiarity} R, Python, SQL, HTML, XML, Shell, \\
{\bf Operating Systems}: Windows, Linux (Ubuntu starting *2008)\\
{\bf Tools}: IBM Cognos Business Intelligence Server, IBM WebSphere Studio, Weka, Vim, Latex, Subversion, Git, Matlab, D3, Tomcat, Gephi, Microsoft Visual Studio \\
{\bf Database}: Oracle, MySql, DB2, MS-Access 
{\bf Hardware used/ built:} IR Sensor kit, 4 wheeled robot, Atmega Circuit, RS 232, LCD, Thermostat.
\section{\sc Roles, Responsibilites \& Accomplishments}
\begin{itemize} %Job Description%
  \setlength{\itemsep}{0.01pt}
  \setlength{\parskip}{0pt}
  \setlength{\parsep}{0pt}
  \item Reviewer for SADHANA'14 {\em Academy Proceedings in Engineering Sciences}, an official journal of Indian Academy of Sciences.
  \item Technical Head for the International Technical Festival TECHNEX 2012 @ IIT(BHU), India.
  \item Donated Blood as a student blood donor in blood donation camp in the year 2010, @IIT Blood Donation Camp affiliated to Sir Sundarlal Hospital, C/o IIT(BHU), India.
  \item Conducted and Organised free workshops for students of every year for 3 continuous years in Robotics under the Robotics Club of  Science and Technology Council, IIT(BHU), India.
  \item Mentored over 30 teams in development of Robotics and Programming Projects, under the banner of Robotics Club of  Science and Technology Council, IIT(BHU), India.
  \item Served the position of Joint Secretary in  Robotics Club of  Science and Technology Council, IIT(BHU), India in the year 2011.
  \item Organiser, Mentor and Lead for the Event Modex 2011 as part of the International Technical Festival TECHNEX 2011 @IIT(BHU), India.
  \item Special Cadet completing {\bf C and B} certification in training by National Cadet Corps, led by the Indian Army Stationed @IIT(BHU), India.{\bf(*C being the highest cadet certificate)} 
\end{itemize}

% \section{\sc Papers}
% %%%%%%%%%%%%%%%%%%%%%%%%%%%%%%%%%%%%%%%%%%%%%%%%%%%%%%%%%%%%%%%%%%%%%%%%%%%%%%%%%%%%%%%%%%%%%%%%%%%%%%%%%%%%%%%


% \section{\sc References }
% Available upon request.

\end{resume}
\end{document}
